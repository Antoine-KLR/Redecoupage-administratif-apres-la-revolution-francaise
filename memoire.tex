\documentclass[a4paper]{book}

%% Language and font encodings
\usepackage[frenchb]{babel}
\usepackage[utf8x]{inputenc}
\usepackage[T1]{fontenc}
\usepackage{lipsum}
%% Sets page size and margins
\usepackage[a4paper,top=3cm,bottom=2cm,left=3cm,right=3cm,marginparwidth=1.75cm]{geometry}

%% Useful packages
\usepackage[dvipsnames]{xcolor}
\definecolor{ubuntu}{RGB}{48,10,36}
\definecolor{stringvert}{RGB}{120,226,52}
\usepackage{float} 
\usepackage{amsmath}
\usepackage{graphicx}
\usepackage{listings}
\usepackage[colorinlistoftodos]{todonotes}
\usepackage[colorlinks=true, allcolors=black]{hyperref}
\title{Mémoire PFE}
\author{EV2 Keller Antoine}




\begin{document}
\begin{titlepage} % Suppresses displaying the page number on the title page and the subsequent page counts as page 1
	\newcommand{\HRule}{\rule{\linewidth}{0.5mm}} % Defines a new command for horizontal lines, change thickness here
	
	\center % Centre everything on the page
	
	%------------------------------------------------
	%	Headings
	%------------------------------------------------
	
	\textsc{\LARGE École navale}\\[0.5cm] 
	\textsc{\LARGE Promotion 2016}\\[0.5cm]
	\textsc{\Large Projet de fin d'étude}\\[0.5cm] 
	\textsc{\large VA SIM}\\[0.5cm] 
	\HRule\\[0.4cm]
	
	{\huge\bfseries  Redécoupage administratif après la révolution française : des paroisses de l’Ancien Régime à celles de la Constitution}\\[0.4cm] % Title of your document
	
	\HRule\\[1.5cm]
	
	%------------------------------------------------
	%	Author(s)
	%------------------------------------------------
	
	\begin{minipage}{0.4\textwidth}
		\begin{flushleft}
			\large
			\textit{EV2}\\
			\textsc{Antoine KELLER}\\% Your name
		\end{flushleft}
	\end{minipage}
	~
	\begin{minipage}{0.4\textwidth}
		\begin{flushright}
			\large
			\textit{Encadrant}\\
			 \textsc{Nathalie ABADIE [Chargée de recherche à IGN - COGIT]} % Supervisor's name
		\end{flushright}
	\end{minipage}
	\vfill\vfill
    
	\includegraphics[width=0.5\textwidth]{logo_EN.jpg}\\ 
\vfill\vfill
\end{titlepage}
	
	 
	%----------------------------------------------------------------------------------------
	
\frontmatter
\section*{Fiche de résumé du projet}

Filière : VA/SIM
\bigbreak
Responsable de la filière : Eric SAUX
\bigbreak
Pilote du projet : Nathalie ABADIE (IGN-COGIT)
\bigbreak

\textbf{\centering Présentation du sujet (objectifs - cahier des charges)}

\fbox{\parbox[c][9cm]{\textwidth}{
Lors de ce stage, nous tâchons de retracer sous forme de graphe rdf l'évolution des paroisses lors de la révolution française.

Pour cela, nous utilisons les archives parlementaires de 1787 à 1860, et plus particulièrement les décrets publiés entre 1787 et 1799 ; ainsi qu'une numérisation de la planche 52 de la carte de Cassini sous un format de base de données spatiales. 


A partir d'un traitement automatique de langage naturel (TALN), nous extrairons les noms des paroisses et la façon dont elles se sont regroupées. Ainsi en déterminant le mouvement des frontières des différentes paroisses, nous pourrons tenter de redessiner les paroisses avant et après la révolution.


Données de référence : 
\begin{itemize}
\item{Tous les toponymes Cassini sur le Puy de Dôme (planche 52 de Cassini)}
\item{Tous les clochers de la carte de Cassini (~ chefs-lieux de paroisses )}
\item{La base des communes de 1804 (EHESS: \href{http://cassini.ehess.fr/cassini/fr/html/1_navigation.php}{lien} )}
\end{itemize}








}}
\vfill


\textbf{\centering Résumé}
\medbreak
\fbox{\parbox[c][9cm]{\textwidth}{



}}
\vfill



%----------------------table des matieres---------------------------------
\newpage
\tableofcontents
%-----------------------------------------------------------------
%--------------table des figures----------------------------
\newpage
\listoffigures

%-------------------------------------------------------------

\newpage
\chapter{Remerciements}




\chapter{Introduction}



 
%--------------------------------------------------------------------------
\mainmatter
\chapter{Contexte historique et objectifs}
\section{L'histoire de la révolution française}


\section{Les décrets de l'Assemblée Nationales relatifs aux paroisses}
\section{l’œuvre de Cassini}


\pagebreak




%--------------------------------------------------------------------------

\newpage
\chapter{État de l'art sur TALN}

Afin de pouvoir repérer et regrouper chacune des paroisses dans les textes de l’assemblée, nous tâcherons d'utiliser un logiciel de traitement du langage. Ce type de logiciel permet d'extraire les mots en les typant, c'est-à-dire en leur assignant une étiquette, généralement au format xml, décrivant le type de mot, sa grammaire, son genre, etc. Ainsi une fois le texte annoté, nous pourrons chercher des patrons de phrases préalablement définis.



\section{Extraction des Entités Nommées}

Une autre approche consiste à extraire les entités nommées d'un texte. Une entité nommée est généralement un nom propre définissant quelque chose d'unique et de concret, et qui appartient à un domaine spécifique (humain, géographique, administratif, etc.). Les entités nommées sont une catégorie d'0entités, telles que définit par les Message Understanding Conferences (MUC) : (Selon \cite{chinchor1998overview}) 
\begin{description}
\item[Named Entities (ENAMEX) :]{ "person", "organization, "location"}
\item[Temporal Expressions (TIMEX) :]{"date", "time"}
\item[Number Expressions(NUMEX) :]{"money", "percent"}
\end{description}

Ces catégories, définies lors de la conférence MUC-6, ont depuis été élargies en catégories bien plus spécifiques, comme le souligne le travail de David Nadeau \cite{nadeau2007survey}. On peut par exemple voir que les types "ville", "pays", "états" font désormais partis de la catégorie "location" ; et que d'autre catégories ont vu le jour : "email adresses", "phone number", "cell type", "DNA", etc.

Les entités peuvent ensuite être agrémentées d'attributs les détaillant.


%%%%%%%%%%%%%%%%%%%%%%%%%%%%%%%%%%%%
Les entités spatiales nommées  sont de deux types : les ESN absolues et les ESN relatives



%%%%%%%%%%%%%%%%%%%%%%%%%%%%%%%%%%%%


La difficulté pour un logiciel réside dans l'identification de ces entités nommées. En effet, comme dit précédemment, le plus souvent ce sont des noms propres, mais il faut pouvoir reconnaître également les synonymes, différencier les homonymes, ainsi que les références faites par métonymies. Par exemple, dans la phrase \textit{"Jean travaille dans la capitale"}, le terme "capitale" doit renvoyer à Paris. 

%%%%%%%%%%%%%%%%%%%%%%%%%%%%%%%%%%%%%%%
\bigbreak

Une approche classique de traitement du langage naturel peut être décrite par la chaîne de traitement suivante : (i) lemmatisation des mots (ii) analyse morpho-lexicale (iii) analyse syntaxique (iv) analyse sémantique. \cite{ABOL2003}

La lemmatisation consiste à retrouver la forme canonique d'un mot, c'est-à-dire la forme du mot dénuée de tout accord ou grammaire, donnée dans un dictionnaire par exemple. L'analyse morpho-lexicale cherche justement à reconnaître la nature et la forme grammaticale employées. Elle permet de définir le genre et le nombre employés pour un nom, la conjugaison et le temps pour un verbe, etc. L'analyse syntaxique permet de lier chacun des mots d'une phrase entre eux, on peut par la suite tracer un graphe où les nœuds seraient les mots et les edges les liens grammaticaux ou syntaxiques (compléments du nom, compléments d'objet, liens de possession). Enfin l'analyse sémantique cherche à donner du sens aux mots, on utilise ici des patrons, ou pattern, de mots (nom, verbe et complément) préalablement définis selon l'application voulue.








\bigbreak



%a refaire
Afin de pouvoir extraire toutes ces entités nommées, nous allons voir maintenant les techniques de désambiguïsation.
 



\section{Extraction des relations spatio-temp}




\section{Techniques de désambiguïsation}


\textit{Notes : }


comparer avec une bdd d'entités nommées (GeoNw, DBpedia, Cassini)

comparer les distances entre deux objets

comparer par rapport à un arbre hiérarchique (monde -> europe -> france -> paris)










\cite{MARTINEAU2007}

Machine learning



70\% des datas sur le net ont des propriétés géographiques d'apres :

C. B. Jones, A. I. Abdelmoty, D. Finch, G. Fu, and
S. Vaid. The spirit spatial search engine: Architecture,
ontologies and spatial indexing. In Geographic
Information Science, Third International Conference,
GIScience, Adelphi, MD, USA, volume 3234 of Lecture
Notes in Computer Science, pages 125–139. Springer,
2004.




\bigbreak

Dans le cadre de l'extraction des noms de paroisses, notre travaille consistera en l'extraction d'entités nommées ; de les typer en précisant s'il s'agit de paroisses, d'églises, de hameaux, etc. ; et enfin de les désambiguïser grâce à deux procédés : comparer avec la base de toponymes de Cassini au moyen d'un algorithme de similitude ; si non similitude, comparer les distances (ne garder que la paroisses la plus proches connues). cf chapitre 3.



%--------------------------------------------------------------------------

\newpage
\chapter{Traitement automatique du langage naturel}






%--------------------------------------------------------------------------
\newpage
\chapter{Visualisation des résultats}





%--------------------------------------------------------------------------
\backmatter
\newpage
\chapter{Conclusion}


% L'analyse de la dynamique des espaces géographiques peut être réalisée en étudiant l'évolution des relations entre les entités géographiques. Ces relations peuvent être stockées et analysées par l'intermédiaire de base de données géographiques (Postgres / Postgis) et/ou de graphe (R-igraph). Ce projet cherche à décrire ces deux approches et à tenter de les comparer.

% Nous avons pu voir au travers de ce projet les différences fondamentales entre une base de données spatiales et un graphe. Nous avons étudier également deux architectures logicielles différentes et deux façons d'obtenir le même résultat d'une fonction essentielle lorsque l'on étudie les réseaux, la fonction degré. Nos résultats montrent que nous n'avons pas pu réellement départager ces deux approches, celles ci ayant des temps de calcul très similaire. Néanmoins, on peut indiquer que R-igraph paraît plus performant dans le cas où les fonctions sont appelées plusieurs fois par l'utilisateur sans mise à jour des données.

% De plus, concernant l'ergonomie et la façon dont l'utilisateur se sert de ces outils, on peut noter que l'utilisation de ligne de commande dans le cas de R-igraph ainsi que la connaissance de la syntaxe des fonctions demandent un investissement plus important de la part de l'utilisateur. Afin d'y remédier, il serait intéressant d'implémenter l'automatisation des différentes actions permettant à R de créer des graphes et de les exploiter grâce notamment à l'utilisation de scriptes python.



%--------------------------------------------------------------------------
\appendix
\newpage
\chapter{Annexes}


\renewcommand{\thefigure}{A.\arabic{figure}}
\setcounter{figure}{0}


% \lstset{
% 	upquote=true,
%     columns=flexible,
%     basicstyle=\ttfamily,
% 	language=python,
%     keywordstyle=\color{blue},
%     commentstyle=\color{gray},
%     frame=single,
%     backgroundcolor=\color{gray!15},
%     rulecolor=\color{gray!15}
% }

% \begin{figure}[h]
% \begin{lstlisting}
% degree <- function(graph, v=V(graph),
%                    mode=c("all", "out", "in", "total"), loops=TRUE,
%                    normalized=FALSE){
  
%   if (!is_igraph(graph)) {
%     stop("Not a graph object")
%   }
%   v <- as.igraph.vs(graph, v)
%   mode <- igraph.match.arg(mode)
%   mode <- switch(mode, "out"=1, "in"=2, "all"=3, "total"=3)
  
%   on.exit( .Call(C_R_igraph_finalizer) )
%   res <- .Call(C_R_igraph_degree, graph, v-1,
%                as.numeric(mode), as.logical(loops))
%   if (normalized) { res <- res / (vcount(graph)-1) }
%   if (igraph_opt("add.vertex.names") && is_named(graph)) {
%     names(res) <- V(graph)$name[v]
%   }
%   res
% }
% \end{lstlisting}
% \caption{\label{codeigraph1}Code source de la fonction "degree" du package \textit{igraph}}
% \end{figure}



%--------------------------------------------------------------------------
\newpage





%--------------------------------------------------------------------------


\newpage
\bibliographystyle{plain-fr}
\bibliography{biblio}




%Del Mondo G, Stell J G, Claramunt C, and Thibaud R 2010 A graph model for spatio-temporal evolution.
%Journal of Universal Computer Science 16: 1452–77

%Le Bot S, Trentesaux A, Garlan T, Berne S, and Chamley H 2000 Influence des tempêtes sur la mobilité des
%dunes tidales dans le détroit du pas-de-calais. Oceanologica Acta 23: 129–41


%Randell D A, Cui Z, and Cohn A G 1992 A spatial logic based on regions and connection. In Proceedings of the
%Third International Conference on Principles of Knowledge Representation and Reasoning (KR’92), Cambridge,
%Massachusetts: 165–76

%Rémy Thibaud, Géraldine Del Mondo, Thierry Garlan, Ariane Mascret and Christophe Carpentier, A Spatio-Temporal Graph Model for Marine Dune Dynamics Analysis and Representation, Transactions in GIS, 2013, 17(5): 742–762

\end{document}
